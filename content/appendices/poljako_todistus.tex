\subsection*{Polynomien jakolause}
\label{tod:poljako}

% Aloin tehdä tätä toista todistusta, mutta en ehtinyt kovin pitkälle. T: Jokke
%
% Tarkastellaan aluksi seuraavia polynomien kertolaskuja, jotka voidaan kaikki tarkistaa käsin:
% \begin{align*}
% (x-1)(x+1) & =x^2-1 \\
% (x-1)(x^2+x+1) & =x^3-1 \\
% (x-1)(x^3+x^2+x+1) & =x^4-1 \\
% & \vdots
% \end{align*}
% Näistä huomataan, että polynomi $x^n-1$ on aina jaollinen polynomilla $x-1$.
% Osamääräksi tulee polynomi $x^{n-1}+x^{n-2}+\cdots+x+1$.
% Tämä voidaan myös todistaa laskemalla tulo
% \begin{align*}
% (x-1)(x^{n-1}+x^{n-2}+\cdots+x+1) & =x(x^{n-1}+x^{n-2}+\cdots+x+1)-1\cdot(x^{n-1}+x^{n-2}+\cdots+x+1) \\
% & =x^n+x^{n-1}+\cdots+x^2+x-x^{n-1}-x^{n-2}-\cdots-x-1 \\
% & =x^n-1.
% \end{align*}
% 
% Nyt voidaan siirtyä varsinaisen jakolauseen todistukseen.
% Olkoon $b$ polynomin $P(x)$ juuri. Tarkoituksena on osoittaa, että $P(x)$ on jaollinen polynomilla $x-b$.

Jakolauseen todistus perustuu polynomien jakoyhtälöön, josta tarkemmin kurssilla 12.

\begin{todistus}
Vaikka lauseke $x-b$ ei olisi polynomin $P(x)$ tekijä, niin lähelle päästään: jos polynomin $Q(x)$ kertoimet valitaan sopivasti, voidaan kirjoittaa
\begin{align*}
P(x)&=(x-b)Q(x)+r,
\end{align*}
missä $r$ on jokin vakio, niin sanottu jakojäännös. Jos nyt $b$ on polynomin $P$ nollakohta, sijoitetaan edelliseen yhtälöön $x=b$, jolloin
\begin{align*}
P(b)&=(b-b)Q(b)+r \quad || \ \ P(b)=0 \\
0&=0+r,
\end{align*}
eli $r=0$, joten $x-b$ on polynomin $P(x)$ tekijä. Jos siis $x=b$ on polynomin $P(x)$ nollakohta, $x-b$ on sen tekijä.
\end{todistus}
