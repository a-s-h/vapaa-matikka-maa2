\section{Lisää tehtäviä (Väliaikainen)}

Tässä on lisää tehtäviä, jotka täydentävät olemassa olevia lukuja.
Ne on sijoitettu erikseen, koska aiempi kokonaisuus on tällä hetkellä
testikäytössä (syyskuu 2013). Testikäytön jälkeen tehtävät arvatenkin
ripotellaan muiden joukkoon.



\begin{tehtavasivu}

\Opensolutionfile{ans}[appendices/lisavastaukset]

\begin{tehtava}
   	Muistikaavat on opittava ulkoa ja niiden käytön tulee automatisoitua.
	Laske siis nämä käyttäen muistikaavoja. Tavoite on kirjoittaa vastaus suoraan ilman välivaiheita. Jos se ei vielä onnistu, yritä selvitä yhdellä välivaiheella.
    \begin{alakohdat}
        \alakohta{$(x+2)^2$}
        \alakohta{$(x-5)^2$}
        \alakohta{$(b+4)(b-4)$}
        \alakohta{$(x-1)^2$}
            \alakohta{$(2x+1)^2$}
            \alakohta{$(1-a)(1+a)$}
            \alakohta{$(x-7)^2$}
            \alakohta{$(y+1)^2$}
            \alakohta{$(x-3y)^2$}
            \alakohta{$(3x-1)^2$}
            \alakohta{$(2x+1)(2x-1)$}
            \alakohta{$(t-2)^2$}
            \alakohta{$(2x+3)^2$}
            \alakohta{$(2-a)^2$}
            \alakohta{$(5x+2)(5x-2)$}
            \alakohta{$(3-c)(3+c)$}
            \alakohta{$(10x-1)^2$}
            \alakohta{$(2a+b)^2$}            
            \alakohta{$(x+6)^2$}                                                                                                                                                        
    \end{alakohdat}
    \begin{vastaus}
        \begin{alakohdat}
            \alakohta{$x^2+4x+4$}
            \alakohta{$x^2-10x+25$}
            \alakohta{$b^2-16$}
            \alakohta{$x^2-2x+1$}
            \alakohta{$4x^2+4x+1$}
            \alakohta{$1-a^2$}
            \alakohta{$x^2-14x+49$}
            \alakohta{$y^2+2y+1$}
            \alakohta{$x^2+6xy+9y^2$}
            \alakohta{$9x^2-6x+1$}
            \alakohta{$4x^2-1$}
            \alakohta{$t^2-4t+4$}
            \alakohta{$4x^2+12x+9$}
            \alakohta{$a^2-4a+4$}
            \alakohta{$25x^2-4$}
            \alakohta{$9-c^2$}                                                                                                                                    
            \alakohta{$100x^2-20x+1$}            
            \alakohta{$4a^2+4ab+b^2$}                                                                                                                                                        
            \alakohta{$x^2+12x+36$}                                                                                                                                                        
        \end{alakohdat}
    \end{vastaus}
\end{tehtava}


\begin{tehtava}
   	Muistikaavan mukaisen lausekkeen tunnistaminen on tärkeää.
Tunnista edellisessä tehtävässä laskemasi muistikaavat ja
   	esitä lausekkeet alkuperäisessä muodossaan tulona.
    \begin{alakohdat}
            \alakohta{$1-a^2$} 
	        \alakohta{$a^2-4a+4$}
            \alakohta{$4x^2+4x+1$}
            \alakohta{$x^2-14x+49$}
            \alakohta{$100x^2-20x+1$}            
            \alakohta{$4a^2+4ab+b^2$}                                                                                                                                                        
            \alakohta{$y^2+2y+1$}
            \alakohta{$x^2+4x+4$}
            \alakohta{$4x^2-1$}
            \alakohta{$t^2-4t+4$}
            \alakohta{$25x^2-4$}
            \alakohta{$b^2-16$}
            \alakohta{$x^2-2x+1$}
            \alakohta{$9-c^2$}                                                                                                                                    
            \alakohta{$x^2+12x+36$}                                                                                                                                                        
            \alakohta{$4x^2+12x+9$}            
            \alakohta{$x^2-10x+25$}
            \alakohta{$x^2+6xy+9y^2$}
            \alakohta{$9x^2-6x+1$}
    \end{alakohdat}
    \begin{vastaus}
        \begin{alakohdat}
            \alakohta{$(1-a)(1+a)$} 
            \alakohta{$(a-2)^2$}
            \alakohta{$(2x+1)^2$}
            \alakohta{$(x-7)^2$}
            \alakohta{$(10x-1)^2$}
            \alakohta{$(2a+b)^2$}   
            \alakohta{$(y+1)^2$}
   		     \alakohta{$(x+2)^2$}
            \alakohta{$(2x+1)(2x-1)$}
             \alakohta{$(t-2)^2$}
            \alakohta{$(5x+2)(5x-2)$}
    	    \alakohta{$(b+4)(b-4)$}
 	       \alakohta{$(x-1)^2$}
            \alakohta{$(3-c)(3+c)$}
            \alakohta{$(x+6)^2$}                                                                                                                                                         
            \alakohta{$(2x+3)^2$}            
  		    \alakohta{$(x-5)^2$}  
            \alakohta{$(x-3y)^2$}
            \alakohta{$(3x-1)^2$}
         \end{alakohdat}
    \end{vastaus}
\end{tehtava}

\newpage

\begin{tehtava}
Sievennä muistikaavan avulla
    \begin{alakohdat}
            \alakohta{$(x^2-1)^2$} 
	        \alakohta{$(a^6+3b^3)^2$}
            \alakohta{$(-12-3x)(12-3x)$}
            \alakohta{$(x+\frac{1}{x})^2$}
    \end{alakohdat}
    \begin{vastaus}
        \begin{alakohdat}
            \alakohta{$x^4-2x^2+1$} 
            \alakohta{$a^{12}+6a^6b^3+9b^6$}
            \alakohta{$-(144-9x^2)=9x^2-144$}
            \alakohta{$x^2+2+\frac{1}{x^2}$}
         \end{alakohdat}
    \end{vastaus}
\end{tehtava}

\begin{tehtava} % toisen asteen yhtälö
	Yllättäviä yhteyksiä:
    \begin{alakohdat}
            \alakohta{Perustele, että $\left(2+\sqrt{3}\right)^{-1}= 2-\sqrt{3}$.} 
	        \alakohta{$\star$ Miten tämä yleistyy?}
    \end{alakohdat}

    \begin{vastaus}
    \begin{alakohdat}
            \alakohta{Tutki lukujen $2+\sqrt{3}$ ja $2-\sqrt{3}$ tuloa.} 
	        \alakohta{Yleisesti $\left(a+\sqrt{a^2-1}\right)^{-1}= a-\sqrt{a^2-1}$}
    \end{alakohdat}
    \end{vastaus}
\end{tehtava}

\begin{tehtava} %Tämä voisi olla tulon merkkisäännön kohdalla?
Osoita, että $x^2+\frac{1}{x^2}\geq 2$, kun $x \neq 0$.
    \begin{vastaus}
     Aloita tiedosta $\left(x-\frac{1}{x}\right)^2 \geq 0$ ja sievennä.
    \end{vastaus}
\end{tehtava}

\begin{tehtava} %Tämä voisi olla tulon merkkisäännön kohdalla?
Osoita, että funktio $f(x)=x^4+3x^2+1$ saa vain positiivisia arvoja.
    \begin{vastaus}
     $x^4\geq 0$ ja $x^2 \geq 0$, joten $f(x) \geq 1$.
    \end{vastaus}
\end{tehtava}

\begin{tehtava} 
$\star$ Osoita, että kun $a \geq 0$ ja $b \geq 0$, pätee \\ $\frac{a+b}{2} \geq \sqrt{ab}$. Milloin yhtäsuuruus on voimassa?
    \begin{vastaus}
     Opastus: Aloita tiedosta $\left(\sqrt{a}-\sqrt{b}\right)^2 \geq 0$ ja sievennä. Yhtäsuuruus pätee, kun $a = b$.
    \end{vastaus}
\end{tehtava}

\paragraph*{Toisen asteen yhtälö}

\begin{tehtava} % toisen asteen yhtälö
Neliön muotoisen taulun sivu on 36 cm. Taululle tehdään tasalevyinen kehys, jonka
nurkat on pyöristettu neljännesympyrän muotoisiksi. Kuinka leveä kehys on, kun sen
pinta-ala on puolet taulun pinta-alasta? (ympyrän pinta-ala on $\pi r^2$.)
    \begin{vastaus}
     $7,7$~cm
    \end{vastaus}
\end{tehtava}

\begin{tehtava} % toisen asteen yhtälö
Ratkaise yhtälö $x - 3 = \frac{1}{x}$.
    \begin{vastaus}
    $x =\frac{3 \pm \sqrt{13}}{2}$
    \end{vastaus}
\end{tehtava}

\begin{tehtava} % toisen asteen yhtälö
On olemassa viisi peräkkäistä positiivista kokonaislukua, joista kolmen
ensimmäisen neliöiden summa on yhtä suuri kuin kahden jälkimmäisen
neliöiden summa. Mitkä luvut ovat kyseessä?
    \begin{vastaus}
		$10^2+11^2+12^2 = 13^2 + 14^2$.
    	jos negatiivisetkin luvut sallittaisiin, $(-2)^2+(-1)^2+0^2 = 1^2 + 2^2$ kävisi 			myös. Löytyykö vastaava $4 + 3$ luvun sarja? Entä pidempi?
    \end{vastaus}
\end{tehtava}

\paragraph*{Toisen asteen epäyhtälö}

\begin{tehtava} % toisen asteen epäyhtälö
Jäätelökioskin päivittäiset kiinteät kulut ovat $400$ euroa. Jokainen jäätelö maksaa
kauppiaalle $0,50$ euroa. Kun jäätelön myyntihinta on $x$ euroa, sitä myydään
$1000 - 200x$ kappaletta. 
\begin{alakohdat}
\alakohta{Millä myyntihinnoilla jäätelön myynti on kannattavaa?}
\alakohta{Millä myyntihinnalla saadaan suurin tuotto? Kuinka suuri?}
\end{alakohdat}
    \begin{vastaus}
		\begin{alakohdat}
		\alakohta{$1,00$ \euro \ $<$ myyntihinta $<$ $4,5$ \euro.}
		\alakohta{$2,75$ \euro, jolloin voitto on $612,50$ \euro. Epäilyttävän hyvä bisnes.}
		\end{alakohdat}
    \end{vastaus}
\end{tehtava}

%\begin{tehtava} % 3. asteen yhtälö! Viilaa lukuja?
%Suorakulmion muotoisen pellinpalan mitat ovat $50~\text{cm} \times 60~\text{cm}$.
%Pellin nurkista leikataan pois neliön muotoiset palat ja lopusta taitellaan laatikko.
%Kuinka isot palat pitää poistaa, jotta laatikot tilavuus olisi 9,0 litraa?
%    \begin{vastaus}
%	hmm
%    \end{vastaus}
%\end{tehtava}


\Closesolutionfile{ans}

\end{tehtavasivu}



\newpage
\subsection*{Lisätehtävien vastaukset}

\begin{vastaussivu}
\begin{Vastaus}{186}
        \begin{alakohdat}
            \alakohta{$x^2+4x+4$}
            \alakohta{$x^2-10x+25$}
            \alakohta{$b^2-16$}
            \alakohta{$x^2-2x+1$}
            \alakohta{$4x^2+4x+1$}
            \alakohta{$1-a^2$}
            \alakohta{$x^2-14x+49$}
            \alakohta{$y^2+2y+1$}
            \alakohta{$x^2+6xy+9y^2$}
            \alakohta{$9x^2-6x+1$}
            \alakohta{$4x^2-1$}
            \alakohta{$t^2-4t+4$}
            \alakohta{$4x^2+12x+9$}
            \alakohta{$a^2-4a+4$}
            \alakohta{$25x^2-4$}
            \alakohta{$9-c^2$}
            \alakohta{$100x^2-20x+1$}
            \alakohta{$4a^2+4ab+b^2$}
            \alakohta{$x^2+12x+36$}
        \end{alakohdat}
    
\end{Vastaus}
\begin{Vastaus}{187}
        \begin{alakohdat}
            \alakohta{$(1-a)(1+a)$}
            \alakohta{$(a-2)^2$}
            \alakohta{$(2x+1)^2$}
            \alakohta{$(x-7)^2$}
            \alakohta{$(10x-1)^2$}
            \alakohta{$(2a+b)^2$}
            \alakohta{$(y+1)^2$}
   		     \alakohta{$(x+2)^2$}
            \alakohta{$(2x+1)(2x-1)$}
             \alakohta{$(t-2)^2$}
            \alakohta{$(5x+2)(5x-2)$}
    	    \alakohta{$(b+4)(b-4)$}
 	       \alakohta{$(x-1)^2$}
            \alakohta{$(3-c)(3+c)$}
            \alakohta{$(x+6)^2$}
            \alakohta{$(2x+3)^2$}
  		    \alakohta{$(x-5)^2$}
            \alakohta{$(x-3y)^2$}
            \alakohta{$(3x-1)^2$}
         \end{alakohdat}
    
\end{Vastaus}
\begin{Vastaus}{188}
        \begin{alakohdat}
            \alakohta{$x^4-2x^2+1$}
            \alakohta{$a^{12}+6a^6b^3+9b^6$}
            \alakohta{$-(144-9x^2)=9x^2-144$}
            \alakohta{$x^2+2+\frac{1}{x^2}$}
         \end{alakohdat}
    
\end{Vastaus}
\begin{Vastaus}{189}
    \begin{alakohdat}
            \alakohta{Tutki lukujen $2+\sqrt{3}$ ja $2-\sqrt{3}$ tuloa.}
	        \alakohta{Yleisesti $\left(a+\sqrt{a^2-1}\right)^{-1}= a-\sqrt{a^2-1}$}
    \end{alakohdat}
    
\end{Vastaus}
\begin{Vastaus}{190}
     Aloita tiedosta $\left(x-\frac{1}{x}\right)^2 \geq 0$ ja sievennä.
    
\end{Vastaus}
\begin{Vastaus}{191}
     $x^4\geq 0$ ja $x^2 \geq 0$, joten $f(x) \geq 1$.
    
\end{Vastaus}
\begin{Vastaus}{192}
     Opastus: Aloita tiedosta $\left(\sqrt{a}-\sqrt{b}\right)^2 \geq 0$ ja sievennä. Yhtäsuuruus pätee, kun $a = b$.
    
\end{Vastaus}
\begin{Vastaus}{193}
     $7,7$~cm
    
\end{Vastaus}
\begin{Vastaus}{194}
    $x =\frac{3 \pm \sqrt{13}}{2}$
    
\end{Vastaus}
\begin{Vastaus}{195}
		$10^2+11^2+12^2 = 13^2 + 14^2$.
    	Jos negatiivisetkin luvut sallittaisiin, $(-2)^2+(-1)^2+0^2 = 1^2 + 2^2$ kävisi 			myös. Löytyykö vastaava $4 + 3$ luvun sarja? Entä pidempi?
    
\end{Vastaus}
\begin{Vastaus}{196}
		\begin{alakohdat}
		\alakohta{$1,00$ \euro \ $<$ myyntihinta $<$ $4,5$ \euro.}
		\alakohta{$2,75$ \euro, jolloin voitto on $612,50$ \euro. }
		% Epäilyttävän hyvä bisnes
		\end{alakohdat}
    
\end{Vastaus}
\begin{Vastaus}{197}
	Luvut ovat $-1, 0$ ja $1$ tai $14, 15$ ja $16$.
    
\end{Vastaus}
\begin{Vastaus}{198}
		\begin{alakohdat}
		\alakohta{$-5<x<0$ tai $3 < x$}
		\alakohta{$x<0$ tai $x = 1$}
		\end{alakohdat}
    
\end{Vastaus}
\begin{Vastaus}{199}
	$P(x)>0$ kun $x > \sqrt{a}$ tai $-\sqrt{a}<x<0$.
    
\end{Vastaus}
\begin{Vastaus}{200}
		\begin{alakohdat}
		\alakohta{$3x$}
		\alakohta{$x^3$}
		\alakohta{$-10x^2$}
		\alakohta{$17x^2$}
		\alakohta{$-3x^5$}
		\end{alakohdat}
    
\end{Vastaus}
\begin{Vastaus}{201}
		\begin{alakohdat}
		\alakohta{$3x^3-x^2+3x$}
		\alakohta{$2x-8$}
		\alakohta{$2x^2+2x+7$}
		\alakohta{$3x^2-21x$}
		\alakohta{$-6x^7+24x^4-2x^2$}
		\end{alakohdat}
    
\end{Vastaus}
\begin{Vastaus}{202}
		\begin{alakohdat}
		\alakohta{$x^2+2x-8$}
		\alakohta{$3a^2-2ab-b^2$}
		\alakohta{$a^2+6a+9$}
		\alakohta{$x^2-2x+1$}
		\alakohta{$16x^2-1$}
		\end{alakohdat}
    
\end{Vastaus}
\begin{Vastaus}{203}
		\begin{alakohdat}
		\alakohta{$x(4x+1)$}
		\alakohta{$5xy(x^2+2y)$}
		\alakohta{$(y+3)(y-3)$}
		\alakohta{$(x-2)^2$}
		\alakohta{$(x^3+10)(x-5)$}
		\end{alakohdat}
    
\end{Vastaus}
\begin{Vastaus}{204}
		$x=3$ tai $x=-2$ tai $x=1$.
    
\end{Vastaus}
\begin{Vastaus}{205}
		Koska $x^6\geq 0$ ja $x^2 \geq 0$. (Parilliset potenssit.)
    
\end{Vastaus}
\begin{Vastaus}{206}
		\begin{alakohdat}
		\alakohta{Opastus: Kerro sulut auki.}
		\alakohta{$(x-y)(x^2+xy+y^2)(x+y)(x^2-xy+y^2)$}
		\end{alakohdat}
    
\end{Vastaus}
\begin{Vastaus}{207}
		\begin{alakohdat}
		\alakohta{$x=0$, $x=-2$ ja $x=1$}
		\alakohta{$x<2$ tai $0<x<1$}
		\alakohta{Vähintään 3. (Kuvaajassa näkyvän alueen
		ulkopuolella voisi olla lisää.)}
		\end{alakohdat}
    
\end{Vastaus}
\begin{Vastaus}{208}
		$8(x-2y)(x+2y)(x^2+y^2+7)$. \\
    Opastus: Älä kerro aluksi sulkuja auki vaan käytä heti muistikaavaa.
    
\end{Vastaus}
\begin{Vastaus}{209}
		\begin{alakohdat}
		\alakohta{$x = \frac{7}{3}$}
		\alakohta{$x=-\frac{5}{6}$}
		\alakohta{$ x= \frac{93}{5}=18\frac{3}{5}$}
		\end{alakohdat}
    
\end{Vastaus}
\begin{Vastaus}{210}
		\begin{alakohdat}
		\alakohta{$2 \leq x \leq 7$}
		\alakohta{$-3 < y \leq 0$}
		\alakohta{$z < 5$}
		\end{alakohdat}
    
\end{Vastaus}
\begin{Vastaus}{211}
		\begin{alakohdat}
		\alakohta{$ x > -2$}
		\alakohta{$x < \frac{5}{2}=2\frac{1}{2}$}
		\alakohta{$x \geq \frac{1}{2}$}
		\end{alakohdat}
    
\end{Vastaus}
\begin{Vastaus}{212}
	7 h tai sitä pidemmissä vuokrissa.
    
\end{Vastaus}
\begin{Vastaus}{213}
        $x \leq \frac{a}{a-2}$, kun $a > 2$ \\
        $x \geq \frac{a}{a-2}$, kun $a < 2$ \\
    $x \in \mathbb{R}$, kun $a = 2$ \\
	
\end{Vastaus}
\begin{Vastaus}{214}
		\begin{alakohdat}
		\alakohta{$x= \pm \sqrt{13}$}
		\alakohta{$x=0$ tai $x=-\frac{2}{5}$}
		\alakohta{$x=-3$ tai $x= \frac{1}{2}$}
		\end{alakohdat}
    
\end{Vastaus}
\begin{Vastaus}{215}
		\begin{alakohdat}
		\alakohta{$ -2 < x < 3 $}
		\alakohta{$x \leq 0$ tai $x \geq 5$}
	\end{alakohdat}
    
\end{Vastaus}
\begin{Vastaus}{216}
	Iät ovat 9 ja 21. Yhtälö on $x(30-x)=189$.
	
\end{Vastaus}
\begin{Vastaus}{217}
		$k = 7 \pm 4 \sqrt{3}$
    
\end{Vastaus}
\begin{Vastaus}{218}
		$c=-40$
    
\end{Vastaus}
\begin{Vastaus}{219}
	$-1 < x < -\frac{4}{7}$ tai $2 < x < 5$. Huomioi, että suorakulmion $B$
    sivut ovat positiiviset vain, kun $-1<x<5$.
    
\end{Vastaus}
\begin{Vastaus}{220}
		$x=0$ tai $x=\sqrt[5]{\frac{5}{2}}$
    
\end{Vastaus}
\begin{Vastaus}{221}
		$x=\pm \sqrt{3}$
    
\end{Vastaus}
\begin{Vastaus}{222}
		\begin{alakohdat}
		\alakohta{esimerkiksi $x^4=-1$ }
		\alakohta{esimerkiksi $x^4(x+1)=0$ }
		\alakohta{mahdotonta}
		\end{alakohdat}
    
\end{Vastaus}
\begin{Vastaus}{223}
		$x=0$ tai $x=\sqrt[5]{\frac{5}{2}}$
    
\end{Vastaus}
\begin{Vastaus}{224}
	Luvut, jotka ovat pienempiä kuin $-1$ ja luvut välillä $]0,1[$.
    
\end{Vastaus}
\begin{Vastaus}{225}
		\begin{alakohdat}
		\alakohta{$x=\pm \sqrt{3}$}
		\alakohta{$x=0$ tai $x\frac{-1 \pm \sqrt{21}}{2}$ }
		\end{alakohdat}
    
\end{Vastaus}
\begin{Vastaus}{226}
		\begin{alakohdat}
		\alakohta{$0<x<5$}
		\alakohta{$0<x\frac{1}{2}$ tai $x<-3$}
		\end{alakohdat}
    
\end{Vastaus}
\begin{Vastaus}{227}
		$\sqrt{12}$~m $\approx 3,46$~m
    
\end{Vastaus}
\begin{Vastaus}{228}
		\begin{alakohdat}
			\alakohta{$0$}
			\alakohta{$(x-y)^2$}
		\end{alakohdat}
    
\end{Vastaus}
\begin{Vastaus}{229}
	$x=\frac{6}{\sqrt[5]{2}-1}$
    
\end{Vastaus}
\begin{Vastaus}{230}
	Sivut ovat $5$ ja $12$.
    
\end{Vastaus}
\begin{Vastaus}{231}
	$x \approx 0,69$, $x \approx 1,86$ tai $x \approx 3,03$.
    
\end{Vastaus}

\end{vastaussivu}