\section{Toisen asteen polynomifunktio}

\qrlinkki{http://opetus.tv/maa/maa2/toisen-asteen-polynomifunktio/}{Opetus.tv: \emph{toisen asteen polynomifunktio} (7:59)}

Toisen asteen polynomifunktio on muotoa
\begin{align*}
ax^2+bx+c,
\end{align*}
missä vakiot $b$ ja $c$ voivat olla mitä tahansa reaalilukuja $(b, \ c \in \R)$ ja $a$ voi olla mikä tahansa reaaliluku, paitsi luku nolla $(a \in \R, \ a \neq 0)$.

Toisen asteen polynomifunktion kuvaajaa nimitetään \termi{paraabeli}{paraabeliksi}. Jos toisen asteen polynomifunktiossa $a < 0$ sanomme sen kuvaajaa
alaspäin aukeavaksi paraabeliksi ja vastaavasti, kun $a > 0$, kuvaajaa nimitetään ylöspäin aukeavaksi paraabeliksi.

% FIXME: kuvat alaspäin ja ylöspäin aukeavista paraabeleista
\begin{luoKuva}{paraabelit}
	kuvaaja.pohja(-5, 5, -5, 5, leveys=7)
	
	kuvaaja.piirra("-3*x**2+3", nimi="$-3x^2+3$")
	kuvaaja.piirra("x**2-4", nimi="$x^2-4$")
\end{luoKuva}

\begin{center}
	\naytaKuva{paraabelit}
\end{center}

\begin{tehtavasivu}

\paragraph*{Opi perusteet}

\begin{tehtava}
  Aukeavatko seuraavat paraabelit ylös- vai alaspäin?
  \begin{alakohdat}
    \alakohta{$4x^2 + 100x - 3$}
    \alakohta{$-x^2 + 1337$}
    \alakohta{$5x^2 - 7x + 5$}
    \alakohta{$-6(-3x^2 + 5)$}
    \alakohta{$-13x(9 - 17x)$}
    \alakohta{$100(1-x^2)$}
  \end{alakohdat}

  \begin{vastaus}
    \begin{alakohdat}
      \alakohta{Ylös}
      \alakohta{Alas}
      \alakohta{Ylös}
      \alakohta{Ylös}
      \alakohta{Ylös}
      \alakohta{Alas}
    \end{alakohdat}
  \end{vastaus}
\end{tehtava}

\end{tehtavasivu}