\section{Tekijöihinjako}

\qrlinkki{http://opetus.tv/maa/maa2/polynomin-jakaminen-tekijoihin/}{Opetus.tv: \emph{polynomin jakaminen tekijöihin} (9:50 ja 5:44)}


%Pitkän matematiikan 1. kurssilla on käsitelty lukujen jakamista tekijöihin.
Esimerkiksi luvun $12$ \termi{tekijä}{tekijät} ovat luvut $1$, $2$, $3$, $4$, $6$ ja $12$. Nämä ovat sellaisia
lukuja, joista saadaan luku $12$ kertomalla ne jollain kokonaisluvulla, tai toisin sanottuna luku $12$ voidaan jakaa
millä tahansa näistä luvuista ilman jakojäännöstä.
%Sanotaan myös, että luvun $12$ \termi{alkutekijä}{alkutekijät} ovat $2$ ja $3$, koska luku $12$ voidaan
%ilmaista niiden tulona ($2\cdot 2\cdot 3 = 2^2\cdot 3 = 12$), mutta näitä tekijöitä ei
%voi enää jakaa pienempiin osatekijöihin. Kokonaisluvun tekijät ovat kokonaislukuja ja alkutekijät alkulukuja.

\begin{esimerkki}
Luku $12$ voidaan kirjoittaa tekijöidensä tulona monella eri tavalla. Esimerkiksi
\begin{align*}
&12 = 2 \cdot 6, \\
&12= 4 \cdot 3 \text{ tai } \\
&12= 2 \cdot 2 \cdot 3.
\end{align*}
\end{esimerkki}

Polynomeja voidaan jakaa vastaavalla tavalla tekijöihin. Polynomien tapauksessa tekijöihinjako tarkoittaa
polynomin esittämistä saman- tai pienempiasteisten polynomien tulona. Aste on aina pienempi, ellei kyse ole pelkästä
vakiokertoimen ottamisesta yhteiseksi tekijäksi.

\begin{esimerkki}
Jaa tekijöihin polynomi $10x^3-20x^2$.
\begin{align*}
& 10x^3-20x^2 \\
=& 10(x^3-2x^2) \ \ \ \ &\emph{otetaan $10$ yhteiseksi tekijäksi} \\
=& 10x^2(x-2) &\emph{otetaan $x^2$ yhteiseksi tekijäksi} \\
\end{align*}
\end{esimerkki}

\begin{esimerkki}
Jaa tekijöihin \quad a) $x^3+x$ \quad b) $3x^2+6x.$

Kun jokaisessa termissä on sama tekijä, se voidaan ottaa yhteiseksi tekijäksi:
\begin{enumerate}[a)]
    \item $x^3+x = x(x^2+1)$
    \item $3x^2+6x = 3x(x+2)$.
\end{enumerate}
\end{esimerkki}

\begin{esimerkki}
Jaa tekijöihin polynomi $5x^3-20x^2+20x$.
\begin{align*}
& 5x^3-20x^2+20x \\
=& 5(x^3-4x^2+4x) \ \ \ \ &\emph{otetaan $5$ yhteiseksi tekijäksi} \\
=& 5x(x^2-4x+4) &\emph{otetaan $x$ yhteiseksi tekijäksi} \\
=& 5x(x^2-2\cdot 2x+2^2) &\emph{sovelletaan muistikaavaa} \\
=& 5x(x-2)^2
\end{align*}
\end{esimerkki}

\begin{esimerkki}
Jaa tekijöihin \quad a) $x^2-4$ \quad b) $x^2+8x+16.$

Jaetaan polynomit tekijöihin hyödyntämällä muistikaavoja
\begin{enumerate}[a)]
    \item $x^2-4 = x^2-2^2 = (x+2)(x-2)$
    \item $x^2+8x+16 = x^2+ 2\cdot 4 \cdot x + 4^2 = (x+4)^2$
\end{enumerate}
\end{esimerkki}

%On suositeltavaa tarkistaa itse, että yllä esitetyt tekijöihinjaot todella toimivat. Polynomien tekijöihinjaon toimivuus
%on helppoa tarkistaa -- täytyy vain laskea väitettyjen tekijöiden tulo ja katsoa, onko se alkuperäinen polynomi. Vaikka
%tarkistus onkin helppoa, tässä vaiheessa ei luultavasti vielä ole selvää, miten tekijöihinjaon voisi saada selville
%-- paitsi toisinaan arvaamalla, mutta tähän kysymykseen vastataan myöhemmin tällä kurssilla.

%Polynomien tekijöihinjako ei ole yksiselitteinen, mutta monesti hyödyllisintä on jakaa polynomi tekijöihin samoin kuin
%esimerkkitapauksissa eli niin, että ensimmäisenä on vakiotermi ja kaikissa muissa tekijäpolynomeissa korkeimman asteen
%termin kerroin on 1.
%
%Esimerkki selkeyttänee asiaa. Polynomi $6x^2+30x+36$ voidaan jakaa tekijöihin vaikkapa seuraavilla tavoilla:
%
%\begin{esimerkki}
%\qquad \\
%\begin{itemize}
%    \item $6(x+2)(x+3)$
%    \item $3(2x+4)(x+3)$
%    \item $3(x+2)(2x+6)$
%    \item $2(3x+6)(x+2)$
%    \item $(6x+12)(x+3)$
%    \item $(\frac12 x+1)(12x+36)$
%\end{itemize}
%\end{esimerkki}
%
%Kaikki nämä tavat ovat ''oikein,'' mutta lähes aina ensimmäinen muoto $6(x+2)(x+3)$ on kätevin.
%
%Toisinaan polynomeille voi löytää tekijöitä soveltamalla joitakin seuraavista keinoista:
%
%\begin{itemize}
%\item Otetaan korkeimman asteen termin kerroin yhteiseksi tekijäksi: \\
%$5x^4+3x^2+x-9 = 5(x^4+\frac{3}{5} x^2+\frac{1}{5} x-\frac{9}{5})$
%\item Otetaan $x$ tai sen potenssi yhteiseksi tekijäksi, jos mahdollista: \\
%$x^5+x^3+3x = x(x^4+x^2+3)$
%$x^7+x^6+5x^4+2x^2 = x^2(x^5+x^4+5x^2+2)$
%\item Sovelletaan muistikaavaa käänteisesti \\
%$x^2-5=x^2-\sqrt{5}^2=(x+\sqrt{5})(x-\sqrt{5})$ \\
%$x^2+8x+16=x^2+2\cdot 4x+4^2=(x+4)^2$ \\
%$x^2+x+\frac14=x^2+2\cdot \frac12 x+(\frac12)^2=(x+\frac12)^2$
%\end{itemize}

Kaikkien polynomien tekijöihinjako ei kuitenkaan näillä menetelmillä onnistu. Myöhemmin tässä kirjassa opitaan, miten toisen asteen polynomin voidaan jakaa tekijöihin nollakohtiensa avulla.

Seuraavassa esimerkissä tekijöihin jako on toteutettu termien ryhmittelyn avulla. Se on joissain tapauksissa näppärä tapa jakaa polynomi tekijöihin, mutta oikean ryhmittelyn keksimiseen ei ole mitään sääntöä.

\begin{esimerkki}
Jaa tekijöihin $x^3+3x^2+x+3$.
\begin{equation*}
x^3+3x^2+x+3=x^2(x+3)+1(x+3)=(x^2+1)(x+3)
\end{equation*}
\end{esimerkki}

\begin{esimerkki}
Jaa tekijöihin $x^{11}+2x^{10}+3x+6$.
\begin{align*}
& x^{11}+2x^{10}+3x+6=x^{10}(x+2)+3(x+2)=(x^{10}+3)(x+2)
\end{align*}
\end{esimerkki}

\begin{tehtavasivu}

\paragraph*{Opi perusteet}

\begin{tehtava}
    Esitä tulona ottamalla yhteinen tekijä.
    \begin{enumerate}[a)]
        \item $2x+6$
        \item $x^2 -4x$
        \item $3x^2 - 6x$
    \end{enumerate}
    \begin{vastaus}
        \begin{enumerate}[a)]
        \item $2(x+3)$
        \item $x(x-4)$
        \item $3x(x-2)$
        \end{enumerate}
    \end{vastaus}
\end{tehtava}

\begin{tehtava}
    Jaa tekijöihin.
    \begin{enumerate}[a)]
        \item $-15^5 +10y$
        \item $x^3y^2 +x^2y^3$
        \item $-4a^3 -2a^2 +2ab$
    \end{enumerate}
    \begin{vastaus}
        \begin{enumerate}[a)]
        \item joko $5(-3x^5 +2y)$ tai $-5(3x^5 -2y)$
        \item $x^2y^2(x+y)$
        \item joko $2a(-2a^2 -a +b)$ tai $-2a(2a^2 +a -b)$
        \end{enumerate}
    \end{vastaus}
\end{tehtava}

\begin{tehtava}
    Jaa tekijöihin.
    \begin{enumerate}[a)]
        \item $x^2+6x+9$
        \item $y^2 - 2y+1$
        \item $x^2 -25$
    \end{enumerate}
    \begin{vastaus}
        \begin{enumerate}[a)]
        \item $(x+3)^2$
        \item $(y-1)^2$
        \item $(x-5)(x+5)$
        \end{enumerate}
    \end{vastaus}
\end{tehtava}

\begin{tehtava}
    Jaa tekijöihin.
    \begin{enumerate}[a)]
        \item $4x^2 +4x +1$
        \item $4x^2 +4x +4$
        \item $9-x^2$
          \end{enumerate}
    \begin{vastaus}
        \begin{enumerate}[a)]
        \item $(2x+1)^2$
        \item $4(x^2 +x +1)$
        \item $(3+x)(3-x)$
        \end{enumerate}
    \end{vastaus}
\end{tehtava}

\begin{tehtava}
    Jaa tekijöihin.
    \begin{enumerate}[a)]
        \item $x^3 +x^2 +x +1$
        \item $a^3 +a^2b +2a +2b$
        \item $4m^5 -2m^3 +2m^2 -1$
    \end{enumerate}
    \begin{vastaus}
        \begin{enumerate}[a)]
        \item $(x^2+1)(x+1)$
        \item $(a^2+2)(a+b)$
        \item $(2m^3 -1)(2m^2 -1)$
        \end{enumerate}
    \end{vastaus}
\end{tehtava}

\paragraph*{Hallitse kokonaisuus}

\begin{tehtava}
    Jaa tekijöihin.
    \begin{enumerate}[a)]
    	\item $x^3 - x$
        \item $x^2 - x + \frac{1}{4} $
        \item $9-x^4$
    \end{enumerate}
    \begin{vastaus}
        \begin{enumerate}[a)]
            \item $x(x-1)^2$
            \item $(x-\frac{1}{2})^2$
            \item $(3+x^2)(3-x^2)$
        \end{enumerate}
    \end{vastaus}
\end{tehtava}

\begin{tehtava}
    Jaa tekijöihin.
    \begin{enumerate}[a)]
    	\item $x^2 -4$
    	\item $x^2 -3$
    	\item $5x^2 -3$
    \end{enumerate}
    \begin{vastaus}
        \begin{enumerate}[a)]
            \item $(x+2)(x-2)$
            \item $(x+\sqrt{3})(x-\sqrt{3})$
            \item $(\sqrt{5}x+\sqrt{3})(\sqrt{5}x-\sqrt{3})$
        \end{enumerate}
    \end{vastaus}
\end{tehtava}

\begin{tehtava}
	Jaa tekijöihin.
	\begin{enumerate}[a)]
		\item $4-x^2$
		\item $16-x^4$
		\item $2-x$
	\end{enumerate}
	\begin{vastaus}
		\begin{enumerate}
			\item $-(x+2)(x-2)$
			\item $-(x^2+4)(x^2-4)$
			\item $-(\sqrt{x}-\sqrt{2})(\sqrt{x}+\sqrt{2})$
		\end{enumerate}
	\end{vastaus}
\end{tehtava}

\paragraph*{Lisää tehtäviä}

\begin{tehtava}
	Jaa tekijöihin. Voit tarvittaessa piirtää kuvaajan
		\begin{enumerate}[a)]
		\item $x-\frac{1}{2}$
		\item  $x^2-x-6$ % \{[ \star ]\}
		\item $(x-4)^2-9$ %  \{[ \star ]\}
	\end{enumerate}
	\begin{vastaus}
		\begin{enumerate}
			\item $\left(\sqrt{x}-\frac{1}{\sqrt{2}}\right)\left(\sqrt{x}+\frac{1}{\sqrt{2}}\right)$
			\item $(x-3)(x+2)$
			\item $(x-1)(x-7)$
		\end{enumerate}
	\end{vastaus}
\end{tehtava}

\end{tehtavasivu}
